\documentclass[12pt,a4paper]{article}
\usepackage{arial}
\usepackage{graphicx}
\usepackage{subfigure}
\usepackage{amssymb,amsmath,amsfonts}
\usepackage[utf8]{inputenc}
\usepackage[spanish]{babel}
\graphicspath{{imagenes/}}
\usepackage{geometry}
\newcommand{\tabitem}{~~\llap{\textbullet}~~}
\usepackage{caption}
%\oddsidemargin = -.3cm
%\textwidth = 17cm
%\textheight = 24cm
%\headsep = 0cm
\renewcommand{\baselinestretch}{1.5}
\geometry{a4paper,left=2cm,right=2cm,top=2cm,bottom=2cm}


 
\begin{document}
\renewcommand{\listtablename}{Indice de tablas}
\renewcommand{\tablename}{Tabla}
\begin{titlepage}

    \begin{figure}[h]
 	\centering
    	{\includegraphics[height=2cm]{SEP_logo}}
    	\hspace{6cm}
    	{\includegraphics[height=2cm]{TecNM}}
    \end{figure}
    
    \centering
    \vspace{1cm}
    {\Large\bfseries Instituto Tecnológico de Ciudad Guzmán\par \vspace{.5cm}}
    {\Large \scshape Departamento de Ingeniería Eléctrica y Electrónica \par \vspace{.5cm}}
    {\Large \scshape Ingeniería Electrónica \par \vspace{.5cm}}
    \begin{figure}[h]
    \centering
     %\includegraphics[height=3cm]{Logo_Ing_Electronica} \hspace{1cm}
     \includegraphics[height=4cm]{logo}
     %\includegraphics[height=3cm]{electronica}
    \end{figure}
    \par \vspace{.2cm}
   
	{\huge Robotica Fija\par\vspace{.5cm}}
	{\Large\bfseries Reporte Brazo Manipulador\\ \par\vspace{1cm}}
	
	\flushright{
	{\large{\itshape Juan José Carranza García }\hspace{1cm}NC. 15290337\par}
	{\large{\itshape José Miguel Benavides Jimenez }\hspace{1cm}NC. 15290334\par}
	{\large{\itshape Alan Escobar Gonzalez }\hspace{1cm}NC. 15290453\par}
	\vspace{1cm}
	\centering
	{\large Catedrático\par
	Dr.Francisco Ochoa Cardenas\par\vspace{1cm}}
	\flushright
	{\large Cd. Guzmán, Jalisco,. 28 de Noviembre del 2018\par}}
\end{titlepage} 

   \newpage
   
   
   \begin{center}
    \textbf{\textit{Introducción}}
   \end{center}
   En el siguiente trabajo se buscara introducir al lector en el amplio mundo de los microcontroladores, se abordaran conceptos tecnicos del microcontrolador asi como la arquitectura con la que estos trabajan.\\
   Se estudiaran los registros internos y se mencionara cual es el nombre y el proposito de cada uno de estos, se dara una introduccion al lenguaje ensamblador y se explicaran los primeros comandos basicos para poder realizar un programa dentro del IDE MPLAB para microcontroladores PIC, y se mostraran algunos simuladores, debugger y emuladores que existen para esta familia de circuitos integrados.\\
   A lo largo de la lectura, se observara la funcionalidad de cada una de las terminales del mirocontrolador pic16f887 , y se veran los encapsulados disponibles para este microcontrolador.
   
   \newpage
   \tableofcontents
   \newpage
   \listoffigures
   \newpage
   \listoftables
   \newpage
   \renewcommand{\figurename}{Figura 1.}
    \renewcommand{\tablename}{Tabla 1.}
   \setcounter{table}{0}
   \section{Objetivos}
 
   \begin{thebibliography}{10}
     \bibitem{Ravi}\textsc{Ravi}, (2017, November 13). <<ELECTRONICS HUB,>> [Online].Available: https://www.electronicshub.org/microcontrollers-basics-structure-applications/.
     
     \bibitem{Carva}\textsc{M. Carvajal}, (2007, September 15). <<Microprocesadores y Microcontroladores,>> [Online].Available: http://microrpocesadores-blog.blogspot.mx/2007/09/diferencia-entre-microprocesadores-y.html.
     
     \bibitem{Furber}\textsc{S. B. Furber}. \textit{VLSI RISC Architecture and Organization}. New York: MARCEL DEKKER,1989.
     
     \bibitem{Camacho}\textsc{R. Camacho}, (2012, April 9). <<RCM computo Integrado,>> [Online].Available: http://rcmcomputointegrado.blogspot.mx/2012/04/arquitectura-von-neumann.html.
     
     \bibitem{Jain}\textsc{P. Jain}, (2012, December 24). <<Enginners Garage,>> [Online].Available: https://www.engineersgarage.com/articles/risc-and-cisc-architecture.
     
      \bibitem{Muha}\textsc{M. A. Mazidi, R. D. McKinlay y C. Danny}. \textit{PIC MICROCONTROLLER AND EMBEDDED SYSTEMS Using Assembly and C for PIC18}. New Jersey: Pearson,2008.
      
      \bibitem{PIC}\textsc{Microchip}. \textit{PIC16f882/883/884/886/887 Data Sheet}. Microchip Technology Inc. ,2009.
      
      \bibitem{CCS}\textsc{G. B. Eduardo}. \textit{Compilador C CCS y Simulador PROTEUS para Microcontroladores PIC}. Barcelona: Marcombo,2009.
      
       \bibitem{K150}\textsc{B. Alejandro}, (2017, September 1). <<¿Cómo usar el Programador de PIC K-150?,>> [Online].Available: https://electrocrea.com/blogs/tutoriales/como-usar-programador-de-pic-k-150.
      
    \end{thebibliography}
\end{document}